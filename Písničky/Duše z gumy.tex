\zvyraznit
\hlavicka[Chinaksi]{Duše z gumy}

\slokaBezCisla{
\GSiroky\DSiroky\EmiSiroky\CmajSedmSiroky\akord{$2\times$}
}

\sloka{
Mám duši \G gumovou a srdce \D ze železa \\
ale když \AmiSedm zavoní mi tvoje \C kombinéza\\
duše se \G vzedme a srdce \D zabuší\\
modravý \AmiSedm obláček zavoní \C ovzduším.\\
Kola se protočí, letíme vpřed\\
patníky lížem, jsou sladké jak med\\
a večer po šichtě jemine panečku\\
všichni tam společně hodíme zpátečku.
}

\refren{
\repetice{\G Mám duši z gumy a \D boky plechový\\
jsem \Emi jenom dopravní pros\CmajSedm tředek kolový\\
\G Mám duši z gumy a \D srdce z ocele\\
přesto ho \AmiSedm miluji, řidiče \CmajSedm přítele.}
}

\sloka{
Vždyť jenom pro tebe můj pane řidiči\\
buší mi motor a vře voda v chladiči\\
vždyť jenom pro tebe, ach, ty můj motorů světe\\
blinkry mi blikají a blatník mi kvete.\\
Po vlídném doteku šoféra prahnu\\
kdykoli bude chtít, vždycky mu zahnu\\
a večer po šichtě jemine panečku\\
společně hodíme zpátečku.
}

\refren{}

\sloka{
\G Až jednou za \D mnoho dní\\
\Emi na kilometru \X posledním\\
já \G hrdě vypustím svou \D duši\\
dík \C žes vždycky jel tak, jak \X se sluší.\\
Až jednou za mnoho dní\\
naposled motor zavrní\\
děkuji, že na naší trase\\
jels jako pán a ne jako prase.
}

\refren{}

\slokaBezCisla{
\G Máááá\D ám duši z \Emi gumy \CmajSedm (Společně hodíme zpátečku).\\
\G Máááá\D ám srdce z \AmiSedm ocele \CmajSedm (Řidiče přítele).\\
\G Máááá\D ám duši z \Emi gumy \CmajSedm (Společně hodíme zpátečku).\\
\G Máááá\D ám srdce z \AmiSedm ocele.\CmajSedmSiroky\GSiroky
}

\begin{center}
	~~\hmatCmajSedm~~
\end{center}