\hlavicka[Petr Kalandra]{Břímě}

\sloka{
\G Včera mě spolkl \Hmi Nazareth, aby \C dnes vypliv ostatky \G mý\\
celou noc hledám místo kam složit vlasy šedivý\\
hej pane nechcete mi říct, kde tu můžu složit hlavu svou\\
né řekl a hned zmizel tam za tou velkou bílou budovou.
}

\refren{
\G Sundej \Hmi z něj to \C břímě \G a nechtěj \Hmi za to \C nic\\
\G dědek ať \Hmi už se \C nedře, hej, hej, héééj \\
nandej mi ho na zá\G da.\HmiSiroky\CSiroky\GSiroky\DSiroky\EmiSiroky\DSiroky\CSiroky
}

\sloka{
Přes rameno ranec svůj, jak rád bych nějaký místo měl\\
kam bych se v klidu schoval a na všechno bych zapomněl\\
v tom vidím Carmen s Ghiou jak po ulici proti jdou\\
Carmen že nemá čas, že si spolu rychle užijou.
}

\refren{
Sundej z nich to břímě a nechtěj za to nic holky ať už se nedřou\dots
}

\sloka{
Když přijdeš k paní Mósrové, tak tam není co bys řek\\
připadáš si jak v automatu ve frontě na párek\\
povídá: \uv{Mladej kmete, a což takhle třeba Nathalii?}\\
už není právě nejmladší a tak zůstaň dneska večer s ní.
}

\refren{
Sundej z ní to břímě a nechtěj za to nic holka ať už se nedře\dots
}

\sloka{
Postarší právník mě nakop a to rovnou do koulí\\
povídá já ti cestu ukážu ale musíš s mým psem ven\\
povídám moment soudče já jsem taky Homo sapiens\\
Abych si trochu ulehčil tu ránu jsem mu navrátil.
}

\refren{
Sundej z něj to břímě a nechtěj za to nic dědek ať už se nedře\dots
}

\sloka{
Na dělový kouli z roku 1835 se vezu někam dolů,\\
je načase mi závidět.\\
Tak letím mezi vás, pozdravuje vás tamten svět\\
že prej se nemusíte vracet, je načase mi závidět
}

\refren{
Sundej ze mě to břímě a nechtěj za to nic ať už se nikdy nedřu\dots nandej si ho na záda.
}
