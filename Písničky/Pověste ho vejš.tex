\hlavicka{Pověste ho vejš}{Michal Tučný}

\noindent\textbf{Předmluva:}\\
\emph{Na dnešek jsem měl divnej sen: slunce pálilo a před salonem stál v prachu dav, v tvářích cejch očekávání.
Uprostřed šibenice z hrubých klád, šerifův pomocník sejmul z hlavy odsouzenci kápi a dav zašuměl překvapením.
I já jsem zašuměl překvapením, ten odsouzenec jsem byl já a šerif četl neúprosným hlasem rozsudek:}

\refrenX{1}{
Pověste ho \Emi vejš, ať se houpá,\\
pověste ho \G vejš, ať má \D dost.\\
Pověste ho \Ami vejš, ať se \Emi houpá,\\
že tu \D byl nezvanej \Emi host.\\
}

\refrenX{2}{
Pověste ho, že byl jinej,\\
že tu s náma dejchal stejnej vzduch.\\
Pověste ho, že byl línej\\
a tak trochu dobrodruh.
}

\sloka{
Pově\Emi ste ho za El Passo,\\
za snídani \G v trávě a lodní \D zvon.\\
Za \Ami to že neoplýval \Emi krásou,\\
že měl \C country rád,\\
že se \Ha uměl smát i \Emi vám.
}


\refrenX{3}{
Nad hla\G vou mi slunce \D pálí\\
konec \Ami můj nic neod\G dá\D lí,\\
do svejch \G snů se dívám z \D dáli.\\
A \Ami do uší mi stále zní\\
\Ha tahle moje píseň poslední.
}

\sloka{
Pověste ho za tu banku,\\
v který zruinoval svůj vklad.\\
Za to že nikdy nevydržel\\
na jednom místě stát.
}

\refrenX{3}{}

\refrenX{1}{}
 
\sloka{
Pověste ho za tu jistou,\\
který nesplnil svůj slib.\\
že byl zarputilým optimistou\\
a tak dělal spoustu chyb.\\
}

\sloka{
Pověste ho, že se koukal,\\
a že hodně jed a hodně pil.\\
že dal přednost jarním loukám\\
a pak se \C oženil a pak se \Ha usadil a \Emi žil\dots
}

\refrenX{1}{
\dots\\
Pověste ho \Emi vejš\ldots\\
Pověste ho \G vejš\ldots\D\\
Pověste ho \Ami vejš\ldots\Emi\\
Pově\D ste ho, ať se h\Emi oupá.
}

