\hlavicka{Síla starejch vín}{Škwor}

kapo na 1. pražci\\

\slokaBezCisla{
\GSiroky{}  \DSiroky{}  \EmiSiroky{}  \CSiroky{}  \EmiSiroky{}  \CSiroky{}  \EmiSiroky{}  \DSiroky{}  \GSiroky{}  \GSiroky{}
}

\sloka{
\Emi Podle nepsanejch \C zákonů hledáme \D  něco, co nám nejspíš \X schází.\\
\Emi Podle prastarejch \C pudů, který nás \D nutěj furt za něčim \X jít.\\
\Emi Do stejný řeky nikdy \C nevstoupíš, klacky \D sám sobě pod nohy \X házíš.\\
\Emi Cos chtěl, máš, stejně \C dojdeš k pozná\D ní.\hspace{1em}\X\\
}

\refren{
Že věci \G  některý jsou \D  neměnný, jak \Emi  síla starejch \C  vín\\
A někdy je \Emi tma kolem \C nás\\
To když \Emi zrovna nejsme stá\D lí \\
Najednou \G  blízký jsou si \D  vzdálený\\
To z \Emi očí kouká \C  nám\\
Tak na chvílí \Emi stůj, \C  vzpomínej\\
No a \Emi přiznej, taky jsme se \D  báli, co bude \G  dál. \X\\
}

\sloka{
\Emi  Každej ctí asi \C jinej zákon a kdo \D ví, co nám odvahu \X dává.\\
\Emi  Každej má asi \C jinej kód a ty sám \D  zřejmě všechno víš \X líp.\\
\Emi  Jak když přesadíš \C starej strom, \D taky to nebude velká \X sláva.\\
\Emi  Cos chtěl máš, stejně \C dojdeš k pozná\D ní.\hspace{1em}\X\\
}

\refren{}