\hlavicka{Dobrák od kosti}{Chinaski}

\sloka{
\D Má milá \A jak ti je, tak \G jak ti je?\ \X\\
\D Jsem ten, kdo \A jednou tvý \G tělo zakryje.\ \CaddDevet\\
Jsem ten, kdo tě jednou oddělá.\\
Potkala's zkrátka koho's neměla.
}

\sloka{
\DSiroky\ASiroky\GSiroky\GSiroky
Jsi budoucí krev v mojí posteli.\\
Jsem ten, kdo tě jednou jistojistě zastřelí.\\
Jsem ten kdo ty tvoje krásný oči jednou zatlačí.\\
Jsi moje všechno a mě to nestačí.
}

\refren{
\D Je to vážně silná \X káva, \A pláč a nebo \X vztek \G nic už s tím \G nenaděláš,\\
\D nech mě jenom \X hádat, \A jak jsi hebká na \X dotek \G krásná a \G nedospělá.
}

\sloka{
Víš, všechno má aspoň malý kaz,\\
jsem ten, kdo ti jednou zlomí vaz.\\
Má milá vždyť mě znáš jsem dobrák od kosti,\\
a ty jsi ta co mi to jednou všechno odpustí.
}

\sloka{
\G Sejde z očí \X sejde z mysli,\\
\A jenom blázen věří \X na nesmysly,\\
\G láska je čaroděj \X a ticho prý léčí,\\
\X ale zákon hovoří \X jasnou řečí.
}

\refren{}

\slokaBezCisla{
\D Má milá \A jak ti je, tak \G jak ti je?\ \X\\
\D Jsem ten, kdo \A jednou tvý \G tělo zakryje.\ \CaddDevet
}

\vspace{-25em}
\noindent\hspace*{27em}\hmat[]{p3,p2,n,n,p3,p3}{\akordhmat{G}}\\
\noindent\hspace*{27em}\hmatCaddDevet

