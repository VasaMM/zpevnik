\hlavicka{Dokud se zpívá}{J. Nohavica}

3/4 takt

\sloka{
\C Z Těšína \Emi vyjíždí \DmiSedm vlaky co \F čtvrthodi\C nu,\quad\EmiSiroky\DmiSedmSiroky\GSedmSiroky\\
\C včera jsem \Emi nespal a \DmiSedm ani dnes \F nespoči\C nu,\quad\EmiSiroky\DmiSedmSiroky\GSedmSiroky\\\F 
\F svatý Med\G ard, můj pa\C tron, ťuká \IC si na \Ami če\hspace{1.3em}\IG lo,\quad\G\\
\F ale dokud se \G zpívá, \F ještě se \G neumře\C lo, \Emi hó\hspace{1.2em}\DmiSedm hó.\hspace{1.5em}\GSedm
}

\sloka{
Ve stánku koupím si housku a slané tyčky,\\
srdce mám pro lásku a hlavu pro písničky,\\
ze školy dobře vím, co by se dělat mělo,\\
ale dokud se zpívá, ještě se neumřelo, hóhó.
}

\sloka{
Do alba jízdenek lepím si další jednu,\\
vyjel jsem před chvílí, konec je v nedohlednu,\\
za oknem míhá se život jak leporelo,\\
ale dokud se zpívá, ještě se neumřelo, hóhó.
}

\sloka{
Stokrát jsem prohloupil a stokrát platil draze,\\
houpe to, houpe to na housenkové dráze,\\
i kdyby supi se slítali na mé tělo,\\
tak dokud se zpívá, ještě se neumřelo.
}

\sloka{
Z Těšína vyjíždí vlaky až na kraj světa,\\
zvedl jsem telefon a ptám se: \uv{Lidi, jste tam?}\\
A z veliké dálky do uší mi zaznělo,\\
že dokud se zpívá, ještě se neumřelo.\\
Že dokud se zpívá ještě se neumřelo.
}

