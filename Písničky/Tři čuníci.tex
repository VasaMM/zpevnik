\hlavicka[Jaromír Nohavica]{Tři čuníci}

\sloka{
\CSiroky\GSiroky\CSiroky\\
\C V řadě za sebou\\
tři čuníci jdou\\
ťapají si v blátě\\
cestou neces\Ami tou.\\
\Dmi Kufry nemají\\
cestu nezna\G jí\\
\Dmi vyšli prostě do světa\\
a \GSedm vesele si zpívají.\\
\C Vi-vi vi-vi ví, ví-vi vi-vi ví, ví-vi vi-vi-ví-ví, vi-vi vi-vi-\Ami ví,\\
\Dmi Ví-vi vi-vi ví, ví-vi vi-vi \G ví, \Dmi ví-vi vi-vi ví-vi vi-vi, \GSedm ví-vi vi-vi-ví
}

\sloka{
Auta jezdí tam, náklaďáky sem,\\
tři čuníci jdou, jdou rovnou za nosem,\\
ušima bimbají, žito křoupají,\\
vyšli prostě do světa a vesele si zpívají: ví ví ví\dots
}

\sloka{
Levá, pravá - teď!, přední, zadní - už!,\\
tři čuníci jdou, jdou jako jeden muž,\\
lidé zírají, důvod neznají,\\
proč ti malí čuníci tak vesele si zpívají: ví ví ví\dots
}

\sloka{
Když kopýtka pálí, když jim dojde dech,\\
sednou ke studánce na vysoký břeh,\\
ušima bimbají, kopýtka máchají,\\
chvilinku si odpočinou, a pak dál se vydají: ví ví ví\dots
}

\sloka{
Když se spustí déšť, roztrhne se mrak,\\
k sobě přitisknou se čumák na čumák,\\
blesky bleskají, kapky pleskají,\\
oni v dešti, nepohodě vesele si zpívají: ví ví ví\dots
}

\sloka{
Za tu spoustu let, co je světem svět,\\
přešli zeměkouli třikrát tam a zpět\\
v řadě za sebou, hele, támhle jdou,\\
pojďme s nima zazpívat si jejich píseň veselou: ví ví ví\dots
}