\hlavicka{Řekni, kde ty kytky jsou}{Marie Rottrová}

\sloka{
\C Řekni, kde ty \Ami kytky jsou, \F co se s nima \G mohlo stát,\\
\C řekni, kde ty \Ami kytky jsou, k\Dmi de mohou \G být,\\
\C dívky je tu \Ami během dne \DSedm otrhaly \G do jedné,\\
\F kdo to kdy \C pochopí, k\Dmi do to kdy \G pocho\C pí?
}

\sloka{
Řekni, kde ty dívky jsou, co se s nima mohlo stát,\\
řekni, kde ty dívky jsou, kde mohou být,\\
muži si je vyhlédli, s sebou domů odvedli,\\
kdo to kdy pochopí, kdo to kdy pochopí?
}

\sloka{
Řekni, kde ti muži jsou, co se s nima mohlo stát,\\
řekni, kde ti muži jsou, kde mohou být,\\
muži v plné polní jdou, do války je zase zvou,\\
kdo to kdy pochopí, kdo to kdy pochopí?
}

\sloka{
A kde jsou ti vojáci, co se s nima mohlo stát,\\
a kde jsou ti vojáci, kde mohou být,\\
řady hrobů v zákrytu, meluzína kvílí tu,\\
kdo to kdy pochopí, kdo to kdy pochopí?
}

\slokaOpakovani{1.}{}