\zvyraznit
\hlavicka{Okoř}

\sloka{
\D Na Okoř je cesta, \X jako žádná ze sta,\\
\ASedm roubená je stroma\D ma.\\
\X Když jdu po ni v létě, \X samoten na světě,\\
\ASedm sotva pletu noha\D ma.\\
\G Na konci té cesty \D trnité\\
\E stojí krčma jako \ASedm hrad.\\
\D Tam zapadli trempi, \X hladoví a sešlí,\\
\ASedm začli sobě noto\D vat.
}

\refren{
\X Na hradě Okoři, \ASedm světla už nehoří,\\
\ID bílá paní \ASedm šla už dávno\ID\ spát.\\
\X Ta měla ve zvyku \ASedm podle svého budíku\\
\ID o půlnoci \ASedm chodit straší\ID vat.\\
\G Od těch dob, co jsou tam \D trampové,\\
\E nesmí z hradu \ASedm pryč.\\
\D A tak dole v podhradí, \ASedm se šerifem dovádí,\\
\ID on ji sebral \ASedm od komnaty\ID\ klíč.
}

\sloka{
Jednoho dne z rána, roznesla se fáma,\\
že byl Okoř vykraden.\\
Nikdo neví do dnes, kdo to tenkrát odnes,\\
nikdo nebyl dopaden.\\
Šerif hrál celou noc mariáš\\
s Bílou paní v kostnici.\\
Místo aby hlídal, zuřivě ji líbal,\\
dostal z toho zimnici.
}

\refren{}