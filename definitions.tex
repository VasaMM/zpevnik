% Predefinovani vypisu nadpisu nazvu pisnicek
\makeatletter
\renewcommand\section{%
    \@startsection {section}{1}{\z@}%
                   {-3.5ex \@plus -1ex \@minus -.2ex}%
                   {.1ex \@plus.1ex}%
                   {}%
}
\makeatother


% Příkaz pro okamžité vkládání řádků do obsahu
% https://tex.stackexchange.com/questions/10291/addtocontents-at-end-of-document-not-getting-written-to-toc-file/10297#10297
\makeatletter
\newcommand{\immaddtocontents}[1]{{%
    \let\protect\@unexpandable@protect
    \immediate\write\@auxout{\noexpand\@writefile{toc}{#1}}%
}}
\makeatother

\newcommand{\addContentSection}[1]{\immaddtocontents{\protect\contentsline{Chapter}{\protect\numberline{}\textbf{\hspace{-1em}#1}}{}{}}}


% Definice,jak se má vysázet název písně a její autor
\newcounter{cisloSloky}
\newcommand{\hlavicka}[2]{
    \section*{\raisebox{-2ex}{\emph{ \huge{#1}}}} \hfill #2\hspace{2px}
    \normalsize
    \vskip 3pt
    \hrule height 3pt
    \vskip 10pt
    \setcounter{cisloSloky}{0}
    \addcontentsline{toc}{subsection}{#1 \texttt{\small (#2)}}
}

% Varianta bez autora
\newcommand{\hlavickaBezAutora}[1]{
    \section*{\raisebox{-2ex}{\emph{ \huge{#1}}}}
    \normalsize
    \vskip 3pt
    \hrule height 3pt
    \vskip 10pt
    \setcounter{cisloSloky}{0}
    \addcontentsline{toc}{subsection}{#1}
}


% Zpusob zapisu nadpisu "Obsah" do seznamu vsech kapitol v obsahu
\newcommand{\nadpisobsah}[1]{
    \begin{center}
        \textbf{\Huge #1}
    \end{center}
}
   
\makeatletter
\renewcommand\tableofcontents{%
     \nadpisobsah{\contentsname}
     \@starttoc{toc}%
     \addContentSection{\hspace*{-3em}}% Proc je to potřeba???
}
\makeatother


% The penalty added to the badness of each line within a paragraph (no associated penalty node)
% Increasing the value makes tex try to have fewer lines in the paragraph.
\interlinepenalty=10000


% Příkaz pro nastavení fontu textu písně
\newcommand{\setTextFont}{\fontencoding{T1}\fontfamily{fav}\selectfont}


% Sloka písně
\newcommand{\sloka}[1]{
    \begin{list}{\textbf{\emph{\refstepcounter{cisloSloky}\thecisloSloky.}}}{\setlength{\leftmargin}{10mm}}
        \setTextFont
        \item #1
    \end{list}
}

% Sloka písně bez čísla
\newcommand{\slokaBezCisla}[1]{
    \begin{list}{}{\setlength{\leftmargin}{10mm}}
        \setTextFont
        \item #1
    \end{list}
}
    
% Opakování sloky (číslo bez textu)
\newcommand{\slokaopakovani}[1]{
    \begin{list}{\textbf{\emph{#1}}}{\setlength{\leftmargin}{10mm}}
        \setTextFont
        \item
    \end{list}
}

% Refrén
\newcommand{\refren}[1]{
    \begin{list}{\textbf{\emph{R:}}}{\setlength{\leftmargin}{10mm} \setlength{\labelwidth}{10mm}}
        \setTextFont
        \item #1
    \end{list}
}

% Refrefén s číslem
\newcommand{\refrenX}[2]{
    \begin{list}{\textbf{\emph{R{#1}:}}}{\setlength{\leftmargin}{10mm} \setlength{\labelwidth}{10mm}}
        \setTextFont
        \item #2
    \end{list}
}

% Repetice
\newcommand{\repetice}[1]{
    \begin{list}{\bf [:}{\setlength{\leftmargin}{5mm} \setlength{\topsep}{-0.3em}}
        \setTextFont
        \item #1 {\bf :]}
    \end{list}
}
