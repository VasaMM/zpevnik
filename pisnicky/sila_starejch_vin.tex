\hlavicka{Síla starejch vín}{Harlej}

kapo na 1. pražci\\

\GSiroky{}  \DSiroky{}  \EmiSiroky{}  \CSiroky{}  \EmiSiroky{}  \CSiroky{}  \EmiSiroky{}  \DSiroky{}  \GSiroky{}  \GSiroky{}  \\

\sloka{
\Emi Podle nepsanejch \C zákonů hledáme \D  něco, co nám nejspíš schází.\\
\Emi Podle prastarejch \C pudů, který nás \D nutěj furt za něčim jít.\\
\Emi Do stejný řeky nikdy \C nevstoupíš, klacky \D sám sobě pod nohy házíš.\\
\Emi Cos chtěl, máš, stejně \C dojdeš k pozná\D ní \\
}

\refren{
Že věci \G  některý jsou \D  neměnný, jak \Emi  síla starejch \C  vín\\
A někdy je \Emi tma kolem \C nás\\
To když \Emi zrovna nejsme stá\D lí \\
Najednou \G  blízký jsou si \D  vzdálený\\
To z \Emi vočí kouká \C  nám\\
Tak na chvílí \Emi stůj, \C  vzpomínej\\
No a \Emi přiznej, taky jsme se \D  báli, co bude \G  dál\\
}

\sloka{
\Emi  Každej ctí asi \C jinej zákon a kdo \D ví, co nám odvahu dává.\\
\Emi  Každej má asi \C jinej kód a ty sám \D  zřejmě všechno víš líp.\\
\Emi  Jak když přesadíš \C starej strom, \D taky to nebude velká sláva.\\
\Emi  Cos chtěl máš, stejně \C dojdeš k pozná\D ní\\
}

\refren{Že věci některý\ldots}