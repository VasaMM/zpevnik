\hlavicka{Dívka s vlasem medovým}{}
 
\sloka{
Chci v\D yprávět vám příběh o jedné dívce s vlasem \G edovým\\
                                                         
ale j\ASedm ak to bylo tenkrát v noci, to se přesně nikdy ned\D ovím}

\sloka{
Já vím jen, že jsem spěchal a řek' si tenkrát: tak si cestu zkrať\\
spodem kolem jezera, kde cestu kříží železniční trať.}

\sloka{
Tam v záři bílých světel vidím dívku u přejezdu stát,\\
asi něco se jí přihodio, je to v jejím obličeji znát.}

\sloka{
\uv{Už dlouho tady čekám a ještě nikdo mi nazastavil,\\
odvezte mě, prosím, domů, je to odtud necelých pět mil.}}

\sloka{
Tvář má bílou jako sníh a paže jako mramor šedavý,\\
   proč stála tam na kolejích tak sama v černé noci, kdopak ví.}

\sloka{
\uv{Tam před tím domem zastavte, já za tátou teď musím domů jít,}\\
   tak zmáčknu klakson, zastavím a čekám, až ji přijdou otevřít.}

\sloka{
Z chodby v nočním šeru vrhá lampa na silnici zář,\\
   \uv{pane, já vám vezu dceru, proč máte tedy zamračenou tvář?}}

\sloka{
Po dívce pohled stáčím a marně kolem rozhlížím se tmou,\\
   vedle mě je místo prázdné stejně jako cesta přede mnou.}

\sloka{
Ten muž se nejdřív diví, zřejmě tomu, co se dovídá,\\
   a zblízka na mě civí, když tichým hlasem ke mě povídá:}

\sloka{
\uv{Já nevím, milý pane, zda je to mýlka nebo krutý žert,\\
    tak nastartujte ten svůj bourák, zmizte odtud, ať vás vezme čert!}}

\sloka{
\uv{Předevčírem moje dcera zabila se, tak se radši ztrať,\\
pět mil odtud u jezera, kde cestu kříží železniční trať!}}

\sloka{
Já \E dodnes tudy jezdím a hledám dívku s vlasem med\A ovým\\
\repetice{ale j\HSedm ak to bylo oné noci, to se snad už nikdy ned\E ovím}}
