\hlavicka{Nějak se vytrácíš má lásko}{Olympic}

\begin{multicols}{2}
\sloka{
\G Nějak se \Emi vytrácíš \C má \D lásko\\
\Emi jak mince rozměněná na šestáky\\
\C A se šestáky \D vytrácím se taky \\
\G Nějak se \Emi vytrácíš \C má \D lásko\\
\Ami Jak průvan který zimomřivě studí\\
\C jak jinovatka \D když se ráno budím
}

\sloka{
Nějak se vytrácíš má lásko\\
tak jako blednou v albech podobenky\\
Tak jako žloutnou staré navštívenky\\
Nějak se vytrácíš má lásko\\
Tak jednoduše jako život kráčí\\
malinko k smíchu malinko k pláči \G 
}


\refren{
\Emi Něco mě \C mra\D zí\\
\G něco ve mně \D zebe\\
\Ami ztrácím kus \C sebe\\
\G ztrácím \D tebe\\
\Emi ztrácím kus \C ves\D míru\\
\G A kousek jis\D toty\\
\Ami A lásku na \C míru\\
\G A lásku pros\D toty\\\\

\G Lásku co \Emi z šestáků \C si hrady \D ráda staví\\
a \C která bůhví \D proč tak často \G ochuraví
}

\sloka{
Nějak se propadáš má lásko\\
a je mi různě, jen mi není príma\\
tam,kde je smutno,tam kde je mi zima.\\
Nějak se propadáš má lásko\\
tam, kde mě mrazí z toho,co nás čeká\\
co čeká zítra na člověka.\\\\

\G Kterému \Emi rozbila se \C křehká \D nádoba\\
je \C to zlé pro \D mne,pro tebe a pro \G oba.
}
\end{multicols}