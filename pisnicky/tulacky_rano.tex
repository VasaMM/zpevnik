\hlavicka{Tulácký Ráno}{}

\begin{center}
~~\hmatDsusDva
\hmatDCtyri~~
\end{center}

\sloka{
\Dmi Posvátný je mi \DsusDva každý ráno,\\
\Ami když ze sna budí šumící \Dmi les.\\
\Dmi A když se zvedám \DsusDva s písničkou známou\\
\Ami a přezky chřestí o skalnatou \Dmi mez.
}

\refren{
\Dmi Tulácký ráno \DsusDva na kemp se snáší,\\
\C za chvíli půjdem toulat se \DmiSiroky{} dál.\DCtyriSiroky{} \DsusDvaSiroky{} \Dmi\\
\Dmi A vodou z říčky \DsusDva oheň se zháší,\\
\C tak zase půjdem toulat se \DmiSiroky{} dál.\DCtyriSiroky{} \DsusDvaSiroky{} \Dmi
}
\sloka{
Posvátný je mi každý večer,\\
když oči k ohni vždy vrací se zpět.\\
Tam mnohý z pánů měl by se kouknout\\
a hned by viděl, jaký chcem svět.
}

\refren{\Dmi Tulácký ráno\ldots}

\sloka{
Posvátný je mi každý slovo,\\
když lesní moudrost a přírodu zná.\\
Bobříků sílu a odvahu touhy,\\
kolik v tom pravdy, však kdo nám ji dá?
}

\refren{\Dmi Tulácký ráno\ldots}