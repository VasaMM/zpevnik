\hlavicka{Toulavej}{}

\begin{multicols}{2}
\sloka{
Někdo \Ami z vás, kdo chutnal \G dálku,\\
jeden \Ami z těch, co rozu\E měj',\\
ať vám \Ami poví, proč mi \G říkaj',\\
proč mi \F říkaj' Toula\Ami vej.\\
\\
Kdo mě \Ami zná a v sále \G sedí,\\
kdo si \Ami myslí: je mu \E hej,\\
tomu \Ami zpívá pro všední \G den,\\
tomu \F zpívá Toula\Ami vej.\\
}

\refren{
\F Sobotní ráno \G mě neuvidí\\
u \GSedm cesty s klukama \C stát.\\
\F Na půdě celta se \G prachem stydí\\
\F a starý songy jsem \G zapomněl hrát,\\
zapomněl \Ami hrát.
}

\sloka{
Někdy v noci je mi smutno,\\
často bejvám doma zlej,\\
malá daň za vaše \uv{umí},\\
kterou splácí Toulavej.\\
\\
Každej měsíc jiná štace,\\
čekáš, kam tě uložej',\\
je to fajn, vždyť přece zpívá,\\
třeba smutně, Toulavej.
}

\refren{Sobotní ráno mě neuvidí\ldots}

\sloka{
Vím, že jednou někdo přijde,\\
tiše pískne: no tak jdem,\\
známí kluci ruku stisknou,\\
řeknou:  vítej, Toulavej.\\
\\
Budou hvězdy jako tenkrát,\\
až tě v očích zabolej',\\
celou noc jim bude zpívat\\
jeden blázen - Toulavej.\\
}

\refren{
Sobotní ráno nám poletí vstříc,\\
budeme u cesty stát,\\
vypráším celtu a můžu vám říct,\\
že na starý songy si vzpomenu rád,\\
vzpomenu rád.
}

\sloka{
Někdo z vás, kdo chutnal dálku,\\
jeden z těch, co rozuměj',\\
ať vám poví, proč mi říkaj',\\
proč mi říkaj' Toulavej.}

\end{multicols}