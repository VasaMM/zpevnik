\hlavicka{Desatero}{Lucie Bílá}

\begin{multicols}{2}
\sloka{
Ten \G text má jenom \Emi deset vět\\
Kdo \G chce, ten může \Emi přemejšlet\\
Však \C né každýho \D jeho obsah \G chytí \D \\
\\
Že \G je to návod \C praktic\D ký\\
A \Emi asi platí na \C vždycky\\
Roz\D hodně má \Ha širší využi\Emi tí
}

\refren{
Halle\C lujah, Halle\Emi lujah \\
Halle\C lujah, Halle\G lu-uu\D uu-u\G jah.
}

\sloka{
Těch deset bodů božích vět\\
By klidně mohlo spasit svět\\
Jenže pořád někde něco vázlo\\
\\
Jak umíme se vymlouvat,\\
Já kdybych moh, tak já bych rád\ldots\\
Jenže jsme už dávno zvyklí na zlo\\
}

\refren{
Hallelujah \ldots
}

\sloka{
A nemusíš znát kopce knih\\
A používat slovo hřích\\
Ber to jako recept na kulajdu\\
\\
A nemusíš stát v kostele\\
A vzývat strážný anděle\\
I když já tam radši občas zajdu\\
}
\refren{
Hallelujah \ldots
}

\sloka{
Tak předně bysme neměli\\
Furt lízt do cizích postelí\\
A lhát a rvát se, i když se to nedá\\
\\
A udávat svý sousedy\\
Mít chuť na jejich obědy\\
Taky je to ostuda, až běda\\
}

\refren{
Hallelujah \ldots
}

\sloka{
A možná si zas vzpomenout\\
Že moc práce je dušežrout\\
A že je dobrý sednout si a zpívat\\
\\
A možná děti naučit\\
Že slabí mají právo žít\\
Vždyť už se na to vážně nedá dívat\\
}
\refren{
Hallelujah \ldots
}
\end{multicols}