\hlavicka{Na kameni kámen}{Nedvědi}


\sloka{
\D Jako suchej starej strom,\\
jako \G vše ničící hrom, jak v poli trá\DSiroky{va}.\GSiroky{} \D \\
\D Připadá mi ten náš svět,\\
plnej \G řečí a čím víc tím líp se \D mám.\GSiroky{} \D 
}

\refren{
\D Budem \Emi o něco se rvát,\\
\Emi až tu \G nezůstane \A stát na \G kameni \D kámen.\GSiroky{} \D \\
\D A jestli \Emi není žadnej bůh,\\
\Emi tak nás \G vezme země \A vzduch no a \G potom \D ámen.\GSiroky{} \D 
}

\sloka{
 A to všecno proto jen,\\
že pár pánů chce mít den bohatších králů.\\
Přes všechna slova co z nich jdou,\\
hrabou pro kuličku svou, jen pro tu svou.\\
}

\refren{Budem o něco\ldots}

\sloka{
Možná jen se mi to zdá\\
a po těžký noci příjde, příjde hezký ráno.\\
Jaký bude nevim sám,\\
taky jsem si zvyk na všechno kolem nás.
}
\refren{Budem o něco\ldots}

\refren{
\D Na, na, ná, na, ná, na, ná,\\
Na, na, \G ná, na, ná, na, ná,   na, na, na, ná, \D ná.
}
