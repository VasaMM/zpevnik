\hlavicka{Pověste ho vejš}{Michal Tučný}

\begin{multicols}{2}
\refrenI{
Pověste ho \Ami vejš, ať se houpá,\\
pověste ho \C vejš, ať má \G dost.\\
Pověste ho \Dmi vejš, ať se \Ami houpá,\\
že tu \G byl nezvanej \Ami host.\\
}

\sloka{
Pověste ho, že byl jinej,\\
že tu s náma dejchal stejnej vzduch.\\
Pověste ho, že byl línej\\
a tak trochu dobrodruh.\\

Pověste ho za El Passo,\\
za snídani v trávě a lodní zvon.\\
Za to že neoplýval krásou,\\
že měl \F country rád,\\
že se \E uměl smát i \Ami vám.
}


\refrenII{
Nad hla\C vou mi slunce \G pálí\\
konec \Dmi můj nic neod\C dá\G lí,\\
do svejch \C snů se dívám z \G dáli.\\
A do \Dmi uší mi stále zní\\
tahle \E moje píseň poslední
}


\sloka{
Pověste ho za tu banku,\\
v který zruinoval svůj vklad.\\
Za to že nikdy nevydržel\\
na jednom místě stát.
}

\refrenII{
Nad hlavou\ldots
}

\refrenI{  Pověste ho vejš \ldots
}
 
\sloka{
Pověste ho za tu jistou,\\
  který nesplnil svůj slib.\\
  že byl zarputilým optimistou\\
  a tak dělal spoustu chyb.\\
\\
  Pověste ho, že se koukal,\\
  a že hodně jed a hodně pil.\\
  že dal přednost jarním loukám\\
  a pak se \F oženil a pak se \E usadil\\
  a \Ami žil.
}

  \refrenI{
\repetice{
Pověste ho vejš\ldots
} 
}
  \end{multicols}

